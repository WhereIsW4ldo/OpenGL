\documentclass[a4paper, 12pt, one column]{article}

%% Language and font encodings. This says how to do hyphenation on end of lines.
\usepackage[english]{babel}
\usepackage[utf8x]{inputenc}
\usepackage[T1]{fontenc}
% \usepackage[clean]{svg}

%% Sets page size and margins. You can edit this to your liking
\usepackage[top=1.3cm, bottom=2.0cm, outer=2.5cm, inner=2.5cm, heightrounded,
marginparwidth=1.5cm, marginparsep=0.4cm, margin=2.5cm]{geometry}

%% Useful packages
\usepackage{graphicx} %allows you to use jpg or png images. PDF is still recommended
\graphicspath{ {./images/} }
\usepackage[colorlinks=False]{hyperref} % add links inside PDF files
\usepackage{amsmath}  % Math fonts
\usepackage{amsfonts} %
\usepackage{amssymb}  %
\usepackage{listings}
\usepackage{xcolor}

\definecolor{codegreen}{rgb}{0,0.6,0}
\definecolor{codegray}{rgb}{0.5,0.5,0.5}
\definecolor{codepurple}{rgb}{0.58,0,0.82}
\definecolor{backcolour}{rgb}{0.95,0.95,0.92}
\lstdefinestyle{mystyle}{ 
    commentstyle=\color{codegreen},
    keywordstyle=\color{magenta},
    numberstyle=\tiny\color{white},
    stringstyle=\color{codepurple},
    basicstyle=\ttfamily\footnotesize,
    breakatwhitespace=false,         
    breaklines=true,                 
    captionpos=b,                    
    keepspaces=true,                 
    numbers=left,                    
    numbersep=0pt,                  
    showspaces=false,                
    showstringspaces=false,
    showtabs=false,                  
    tabsize=1
}

\lstset{style=mystyle}

%% Citation package
\usepackage[authoryear]{natbib}
\bibliographystyle{abbrvnat}
\setcitestyle{authoryear,open={(},close={)}}


\title{Computergrafieken: Kraan}
\author{Waldo Wautelet}

\begin{document}
\maketitle

\section{Input-toetsen}

\begin{itemize}
    \setlength\itemsep{0em}
    \item [g]: animatie last (horizontaal)
    \item [G]: animatie last (verticaal)
    \item [r]: animatie roteren
    \item [p]: algemeen perspectief mode
    \item [i]: specifiek perspectief mode
    \item [o]: orthogonaal mode
    \item [n]: een extra kraan tekenen
    \item [N]: een kraan verwijderen (indien meer dan 1)
    \item [x]: plus 1 op x-coordinaat
    \item [X]: min  1 op x-coordinaat
    \item [y]: plus 1 op y-coordinaat
    \item [Y]: min  1 op y-coordinaat
    \item [z]: plus 1 op z-coordinaat
    \item [Z]: min  1 op z-coordinaat
    \item [j]: laat de x, y, z-as zien
    \item [l]: laat draad zien in bezier vlakken
    \item [a]: ambient wit licht aan/uit zetten
    \item [b]: diffuse groen-blauw licht aan/uit zetten
    \item [c]: rood speculair licht aan/uit zetten
    \item [d]: geel spot aan/uit zetten
    \item [v]: spothoek vergroten
    \item [V]: spothoek verkleinen
    \item [w]: exponent spot vergroten
    \item [W]: exponent spot verkleinen
    \item [e]: shininess vergroten met 5
    \item [E]: shininess verkleinen met 5
    \item [h]: spot hoger plaatsen (op y-as)
    \item [H]: spot lager plaatsen (op y-as)
    \item [k]: controlepunten bezier-vlakken aan/uit zetten
    \item [s]: smooth shading aan/uit zetten
    \item [f]: doorzichtigheid beziervlak aan/uit zetten
    \item [m]: mist aan/uit zetten
    \item [M]: exponent/linear aan zetten (mist)
    \item [default]: alle nummertjes tussen 0-9 laten elk enkel een deel zien van de kraan
\end{itemize}

Ik heb bij de input een bewuste keuze gemaakt wegens een mogelijks tekort aan toetsen.
Ik heb gekozen om gebruik te maken van booleans om sommige variabelen te behandelen 
zoals bijv. de animaties. Hierdoor moet ik maar 1 toets gebruiken om aan en uit te schakelen.
Deze methode grbruik ik bij volgende variabelen:

\begin{itemize}
    \item animatie last (zowel horizontaal als verticaal)
    \item animatie rotatie
    \item draad laten zien
    \item alle lichten
    \item controlepunten laten zien
    \item smooth shading
    \item doorzichtigheid
    \item alles rond mist
    \item alle onderdelen apart laten zien
\end{itemize}

Bij het ingeven van de displaymodussen heb ik gebruik gemaakt van 2 coordinaten, omdat de frustum
methode rare coordinaten gebruikt die ik niet zomaar kon gebruiken van de andere. 

\section{Ontwikkelde methodes}

In deze sectie wordt beschreven welke methodes er zijn en wat welke doet.

\begin{lstlisting}[language=C]
    void init(void);
\end{lstlisting}
In de init functie zit alle standaard code in die 1 keer moet uitgevoerd worden 
en daarna niet meer. Hier zit een clear in van het scherm, de camera word geplaatst
en alle lichten krijgen kun kleuren.

\begin{lstlisting}[language=C]
    void winReshapeFcn(GLint w, GLint h);
\end{lstlisting}
Deze functie word opgeroepen wanneer het venster verandert van grootte, hierin word
een keuze gemaakt welk perspectief we willen gebruiken.

\begin{lstlisting}[language=C]
    void displayFcn(void)
\end{lstlisting}
Deze functie tekent alle onderdelen, plaatst de lichten (indien nodig) en tekent mist indien aangezet.
Hierin wordt ook drawKraan opgeroepen (indien nodig meerdere keren met een translatie ertussen).

\begin{lstlisting}[language=C]
    void drawKraan();
\end{lstlisting}
De drawKraan functie tekent alle delen van de kraan indien deze aan staan.

\begin{lstlisting}[language=C]
    int main(int argc, char* argv[]);
\end{lstlisting}
De main functie bevat alle standaard initialiseringen voor openGL, ook word er een timer functie 
doorgegeven en ten slotte nog een toetsenbord functie.

\begin{lstlisting}[language=C]
    void toetsPress(unsigned char toets, int x_muis, int y_muis);
\end{lstlisting}
Alle opties van toetsinvoer worden doorlopen en indien nodig aangepast, ik maak hier gebruik van 
een switch.

\begin{lstlisting}[language=C]
    void drawBalk();
\end{lstlisting}
Deze functie tekent enkel de 4 verticale balken als een gevulde kubus en een lijntjes.

\begin{lstlisting}[language=C]
    void drawAssen();
\end{lstlisting}
Een functie die de 3 hoofdassen tekent in elk een ander kleurtje.

\begin{lstlisting}[language=C]
    void drawVakwerk(char vlak);
\end{lstlisting}




\bibliography{refs}
\end{document}